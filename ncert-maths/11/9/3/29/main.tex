\iffalse
\let\negmedspace\undefined
\let\negthickspace\undefined
\documentclass[journal,12pt,twocolumn]{IEEEtran}
\usepackage{cite}
\usepackage{amsmath,amssymb,amsfonts}
\usepackage{graphicx}
\usepackage{textcomp}
\usepackage{xcolor}
\usepackage{txfonts}
\usepackage{listings}
\usepackage{enumitem}
\usepackage{mathtools}
\usepackage{gensymb}
\usepackage{comment}
\usepackage[breaklinks=true]{hyperref}
\usepackage{tkz-euclide} 
\usepackage{listings}
\usepackage{gvv}                                        
\def\inputGnumericTable{}                                 
\usepackage[latin1]{inputenc}                                
\usepackage{color}                                            
\usepackage{array}                                            
\usepackage{longtable}                                       
\usepackage{calc}                                             
\usepackage{multirow}                                         
\usepackage{hhline}                                           
\usepackage{ifthen}                                           
\usepackage{lscape}
\usepackage[export]{adjustbox}

\newtheorem{theorem}{Theorem}[section]
\newtheorem{problem}{Problem}
\newtheorem{proposition}{Proposition}[section]
\newtheorem{lemma}{Lemma}[section]
\newtheorem{corollary}[theorem]{Corollary}
\newtheorem{example}{Example}[section]
\newtheorem{definition}[problem]{Definition}
\newcommand{\BEQA}{\begin{eqnarray}}
\newcommand{\EEQA}{\end{eqnarray}}
\newcommand{\define}{\stackrel{\triangle}{=}}
\newtheorem{rem}{Remark}

\begin{document}
\parindent 0px
\bibliographystyle{IEEEtran}

\vspace{3cm}
\title{}
\author{EE23BTECH11042 -  Khusinadha Naik$^{*}$
}
\maketitle
\newpage
\bigskip

% \renewcommand{\thefigure}{\theenumi}
% \renewcommand{\thetable}{\theenumi}


\section*{Exercise 9.3}

\noindent \textbf{29.} \hspace{2pt}If A and G be A.M. and G.M., respectively between two positive numbers, prove that the numbers are A $\pm \sqrt{(A+G)(A-G)}$. 

\noindent \textbf{Ans.}\\
\fi

\begin{table}[h]
\centering
\begin{tabular}{|c|c|c|}
        \hline
        \textbf{Parameter} & \textbf{Value} & \textbf{Description} \\
        \hline
        $x_1\brak{n}$ & $\brak{x_1\brak{0}+nd}u\brak{n}$ & AP series \\
	\hline
	$x_2\brak{n}$ & $\brak{x_2\brak{0}\cdot r^{n}}u\brak{n}$ & GP series \\
        \hline
        $x_1\brak{0}, x_2\brak{0}$ & a & First number \\
        \hline
	$x_1\brak{2}, x_2\brak{2}$ & b & Second number \\
	\hline
        $x_1\brak{1}$ & $\brak{x_1\brak{0} + d}u\brak{n} $ & A.M.\brak{A} \\
        \hline
        $x_2\brak{1}$ & $\brak{x_1\brak{0}\cdot r}u\brak{n} $ & G.M.\brak{B} \\
        \hline
\end{tabular}
\caption{Input parameters table}
\label{tab:11.9.3.29.1}

\end{table}

\noindent From \tabref{tab:11.9.3.29.1}
\begin{align}
x_1\brak{2} &= x_2\brak{2} \\
\brak{x_1\brak{0} + 2d }u\brak{n} &= \brak{x_1\brak{0} r^{2}}u\brak{n} \\
2d &= x_1\brak{0}\brak{r^{2} - 1}  \label{eq:11.9.3.29.3}
\end{align}

Now the two numbers are 
\begin{align}
	\brak{a,b} &= \brak{x_1\brak{0}u\brak{n} , \brak{x_1\brak{0} + d} u\brak{n}} \\
&=\brak{x_1\brak{0} + d \pm 2d} u\brak{n} \\
&=\brak{x_1\brak{0} + d \pm \sqrt{d^{2}}} u\brak{n} \\
\notag &=\brak{x_1\brak{0} + d } u\brak{n}\pm\\
& \quad  \sqrt{2x_1\brak{0}d - 2x_1\brak{0}d + d^{2}}  u\brak{n} \label{eq:11.9.3.29.7}
\end{align}

Substituting \eqref{eq:11.9.3.29.3} in \eqref{eq:11.9.3.29.7}:
\begin{align}
\notag (a,b) &= \brak{x_1(0) + d } u\brak{n}\pm \\ 
& \quad \sqrt{2x_1(0)d + x_1(0)  x_1(0)  (1 - r^{2}) + d^{2}} u\brak{n} \\
\notag &= \brak{x_1(0) + d } u\brak{n}\pm \\ 
& \quad \sqrt{x_1(0)^2 + d^2 + 2x_1(0)d - x_1(0)^2r^2} u\brak{n} \\
\notag &= \brak{x_1(0) + d } u\brak{n}\pm \\
& \quad \sqrt{ \brak{\brak{x_1\brak{0} + d} u\brak{n}}^2 - \brak{\brak{x_1\brak{0} r}u\brak{n}}^2}\label{eq:11.9.3.29.10}
\end{align}


\noindent Comparing \eqref{eq:11.9.3.29.10} , \tabref{tab:11.9.3.29.1} 
\begin{align}
\brak{a,b} =& A \pm \sqrt{A^2 - G^2} \\
\implies \brak{a,b} =& A \pm \sqrt{\brak{A + G}\brak{A - G}}
\end{align}

\begin{align}
u\brak{n} \xleftrightarrow{\mathcal{Z}} &\frac{1}{1 - z^{-1}} \quad , \abs{z} > 1  \label{eq:11.9.3.29.13}\\
nu\brak{n} \xleftrightarrow{\mathcal{Z}} &\frac{z^{-1}}{\brak{1 - z^{-1}}^2} \quad , \abs{z} > 1  \label{eq:11.9.3.29.14}\\
a^{n}u\brak{n} \xleftrightarrow{\mathcal{Z}} &\frac{1}{1 - az^{-1}} \quad , \abs{z} > a \label{eq:11.9.3.29.15}
\end{align}

From  \eqref{eq:11.9.3.29.13} , \eqref{eq:11.9.3.29.14}
\begin{align}
x_1\brak{z} = \frac{x\brak{0}}{1 - z^{-1}} + \frac{dz^{-1}}{\brak{1 - z^{-1}}^2} \quad , \abs{z} > 1 
\end{align}
From \eqref{eq:11.9.3.29.15}
\begin{align}
x_2\brak{z} = \frac{x\brak{0}}{1 - rz^{-1}} \quad , \abs{z} > r
\end{align}




%\end{document}
